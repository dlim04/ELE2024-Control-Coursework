\subsection*{Problem A5} 
    \hfill \break \\
Apply Laplace Transform to Equations \eqref{eq:30},\eqref{eq:31} and \eqref{eq:32}:
\begin{align}
    s\bar{X_1} &= \bar{X_2} \label{eq:33}\\
    s\bar{X_2} &= a\bar{I} + b\bar{X_1} - c\bar{X_2}\label{eq:34} \\
    s\bar{I} &= \frac{\bar{V}}{f} - \frac{\bar{I}R}{f}\label{eq:35}
\end{align}
Sub Equation \eqref{eq:33} into \eqref{eq:34}:\\
\begin{align}
     s(s\bar{X_1}) &= a\bar{I} + b\bar{X_1} - c(s\bar{X_1})\nonumber \\
     s^2\bar{X_1} &= a\bar{I} + b\bar{X_1} - cs\bar{X_1}\nonumber \\
     a\bar{I} &= s^2\bar{X_1} - b\bar{X_1} + cs\bar{X_1}   \label{eq:36}
\end{align}
Rearrange Equation \eqref{eq:35}:\\
\begin{align}
     s\bar{I} = \frac{\bar{V}}{f} - \frac{\bar{I}R}{f}\nonumber \\
     sf\bar{I}+ \bar{I}R = \bar{V} \nonumber \\
     \bar{I}(sf+ R) = \bar{V} \nonumber \\
     \bar{I} = \frac{\bar{V}}{(sf+ R)} \label{eq:37}
\end{align}
Sub Equation \eqref{eq:37} into \eqref{eq:36}:\\
\begin{align}
    s^2\bar{X_1} - b\bar{X_1} + cs\bar{X_1} = a\frac{\bar{V}}{(sf+ R)} \nonumber \\
     (s^2 - b + cs)\bar{X_1} = \bar{V}\frac{a}{(sf+ R)} \label{eq:38}
\end{align}
The Transfer Function for Equation \eqref{eq:38} where $\bar{V}$ is the input,voltage across the circuit and $\bar{X_1}$ is the output of the system, the position of the ball on the inclined plane. 
\begin{equation}\label{eq:39}
G(s) = \frac{\bar{X_1}(s)}{\bar{V}(s)} = \frac{a}{(sf+R)(s^2 - b + cs)}
\end{equation}
Poles of Equation \eqref{eq:39}:
\begin{equation} \nonumber
    \frac{a}{(sf+R)(s^2 - b + cs)} = 0
\end{equation}
\begin{align}
    sf+R = 0 \nonumber \\
    \therfore s = -\frac{R}{f} \label{eq:40}
\end{align}
\begin{align}
    s^2 - b + cs = 0 \nonumber \\
    s = \frac{-c \pm \sqrt{c^2 - 4(1)(-b)}}{2} \nonumber \\
    = \frac{-c \pm \sqrt{c^2 + 4b}}{2} \nonumber \\
    \therfore s = \frac{-c + \sqrt{c^2 + 4b}}{2} ,\frac{-c - \sqrt{c^2 + 4b}}{2} \label{eq:41}   
\end{align}
As shown in Equations \eqref{eq:40} \&\ \eqref{eq:41}, Equation \eqref{eq:39} has 3 poles. Furthermore the presence of complex poles indicates that the impulse response is oscillatory.
\pagebreak